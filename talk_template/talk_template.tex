%\documentclass[11pt]{beamer}
%%
%% Choose how your presentation looks.
%%
%% For more themes, color themes and font themes, see:
%% http://deic.uab.es/~iblanes/beamer_gallery/index_by_theme.html
%%
%
%
%\mode<presentation>
%{
%  \usetheme{Madrid}      % or try Darmstadt, Madrid, Warsaw, ...
%  \usecolortheme{default} % or try albatross, beaver, crane, ...
%  \usefonttheme{default}  % or try serif, structurebold, ...
%  \setbeamertemplate{navigation symbols}{}
%  \setbeamertemplate{caption}[numbered]
%
%} 
%\usepackage{amsmath,amssymb,amsthm,graphicx,float,url,subfigure,cite, pdflscape,color,setspace, xspace}
%\usepackage{multimedia}
%\usepackage[english]{babel}
%\usepackage[utf8x]{inputenc}
%%\usepackage{media9}

\documentclass[aspectratio=43]{beamer}

\usepackage[utf8]{inputenc}
\usepackage[english]{babel}
\usepackage{csquotes}%Correctly typeset Vojtech
\usepackage{csvsimple}
\usepackage{booktabs}
\usepackage{amsmath}
\usepackage{amsfonts}
\usepackage{amssymb}
\usepackage{mathtools}
\usepackage{color}
\usepackage{listings}
\usepackage{styles/beamer_eit-en}
\usepackage[export]{adjustbox}
\usepackage[font={scriptsize}]{caption}%Smaller image captions, it for italics, can choose footnotesize to be even smaller
\captionsetup[figure]{labelformat=empty}%Remove "Figure" from the figure caption
%\setbeamertemplate{footline}[frame number]
%\setbeamerfont{footline}{series=\bfseries}

%Mathematic packages
\usepackage{amsthm,graphicx,float,url,subfigure,cite, pdflscape,color,setspace, xspace}
\usepackage{amsbsy}
\usepackage{relsize}
\definecolor{backcolour}{rgb}{0.95,0.95,0.92}
\lstdefinestyle{customc}{
  backgroundcolor=\color{backcolour},
  belowcaptionskip=1\baselineskip,
 breakatwhitespace=false,
    breaklines=true,
    captionpos=b,
    keepspaces=true,
    numbersep=5pt,
    showspaces=false,
    showstringspaces=false,
    showtabs=false,
    tabsize=2
  basicstyle=\footnote[frame]size\ttfamily,
  keywordstyle=\bfseries\color{green!40!black},
  commentstyle=\itshape\color{purple!40!black},
  identifierstyle=\color{blue},
  stringstyle=\color{orange},
}
%\lstset{escapechar=@,style=customc}

\usepackage{xcolor}


\newcommand{\red}[1]{{\color{red}#1}}

\title[Shortened title]
{GERC/MultiForm Talk template:\\
{\smaller Subtitle}
}
\author[Shortened author names]{Authors}
\institute{University of Nottingham}
\date{\today}

\begin{document}

\begin{frame}
  \titlepage
\end{frame}

% Uncomment these lines for an automatically generated outline.
%\begin{frame}{Outline}
% \tableofcontents
%\end{frame}

%%%%%% Matteo - overview of research at UoN and overall theme of talk
\begin{frame}{Research @ University of Nottingham}


{\begin{center}
\includegraphics[width=0.5\linewidth]{GERC-logo.jpg}
\end{center}
}


\begin{itemize}
\item Interdisciplinary research centre on porous media energy applications
\item \textbf{Mathematics}, Engineering, Chemistry
\item Applications: Carbon Capture and Storage, Unconventional reservoirs, Geothermal, Li-ion batteries
\item Pore-scale modelling, Upscaling, Porous media characterisation
\end{itemize}
\end{frame}



\begin{frame}{Multiscale Fluid Dynamics and Porous Media Group}{}
\begin{center}
{\Large
\url{www.multiform.xyz}
}
\end{center}

\begin{itemize}
\item Non-Fickian transport and mixing\footnote[frame]{DENTZ M., HIDALGO J., ICARDI M., Mechanisms of Dispersion in Porous Media. Journal of Fluid Mechanics, 2018.}
\item \alert{Heterogeneous reactions, adsorption, deposition}\footnote[frame]{BOCCARDO G., CREVACORE E., SETHI R., ICARDI M., A robust upscaling of the effective particle deposition rate in porous media. Journal of Contaminant Hydrology, 2018.}
\item Conjugate heat/mass transfer\footnote[frame]{MUNICCHI F., ICARDI M., Generalised Multi-Rate Models for conjugate transfer in heterogeneous materials, arXiv preprint}
\item Particulate flows\footnote[frame]{ICARDI M., NIASAR V., SCHREYER L., Coupled processes in charged porous media - from theory to applications, Transport in Porous Media. 2019.}, suspensions
\end{itemize}
\end{frame}


%%%%%% Emma - Motivation behind pore-scale simulations and OpenFOAM implementation

\begin{frame}{Governing equations}
Transport of concentration field in the fluid, $u$, throughout the porous media with binding of solute to the surface, $s$:
\begin{align*}
&\frac{\partial u}{\partial t} + \underline{v}\cdot\left(\triangledown u\right) - D_{m}\triangledown^{2}u = 0, \\
%&s_{t}=K_{a} f_{i}\left(\frac{s}{s_{ref}}\right) u - K_{d} f_{j}\left(\frac{u}{u_{ref}}\right) s.
\label{eq:advdiff_pore}
\end{align*}
\end{frame}

\begin{frame}{Governing equations}
The boundary condition for the reactive surface is given by:
\begin{equation*}
-D \nabla_n u = \underbrace{K_{a} f_{i}\left(\frac{s}{s_{ref}}\right) u}_{adsorption} - \underbrace{K_{d} f_{j}\left(\frac{u}{u_{ref}}\right) s}_{desorption}
= \frac{\partial s}{\partial t}.
\label{eq:BC_gen_f}
\end{equation*}
\end{frame}

\begin{frame}{Example illustrations}
To center the figures and make them larger than the space allowed for text use the command \\makebox
\begin{figure}
\makebox[\textwidth][c]
{
\includegraphics[width=1.3\linewidth]{GERC-logo}
}
\end{figure}
\end{frame}


\begin{frame}{Conclusions}
\begin{itemize}
\item something
\item something else
\item etc... 
\end{itemize}
Future work:
\begin{itemize}
\item improving these templates
\item etc...
\end{itemize}

\end{frame}

\end{document}

